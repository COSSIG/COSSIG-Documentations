% 这个文件包含了 COSSIG 贡献者指南的 Latex 源代码。
% 由于主要维护者(Northurland)tex 不熟练,目前不推荐贡献这个项目。
% 不过如果你能教教我……那最好了!

% 这个文件包含了合规的文档样式。如果想要让你的文档的样式符合规定,请在你的 tex 代码中引用这个文件。

\documentclass{article}[10pt, a4paper]

\usepackage{geometry}
[
    left=25mm, right=25mm, top=25mm, bottom=25mm, headheight=18pt
]

\usepackage{setspace}
\usepackage{hyphenat}
\usepackage{datetime2}
\usepackage{fontspec}
\usepackage{parskip}
\usepackage[x11names]{xcolor}
\usepackage[normalem]{ulem}
\usepackage{titling}

\usepackage{xeCJK}
\xeCJKsetup{CJKspace=true}  % 编译时不忽视中、日、韩文文字间的空格

% 正文行间距设定
\doublespacing{}

% 正文字体 - “思源”系列
\setCJKmainfont{Source Han Sans SC}  % 中文字体:思源黑体(大陆简体)
\setmainfont{Source Sans Pro}
\setmonofont{Source Code Pro}

% 特殊用途字体 - 用于序号和特别标题

\newfontfamily\Montserrat{Montserrat-Regular}
[
    ItalicFont = Montserrat-Italic,
    BoldFont = Montserrat-Bold,
    BoldItalicFont = Montserrat-BoldItalic
]

\newfontfamily\MontserratExb{Montserrat-ExtraBold}
[
    ItalicFont = Montserrat-ExtraBoldItalic,
    BoldFont = Montserrat-Black,
    BoldItalicFont = Montserrat-BlackItalic
]

% 标题设定
% 左对齐标题;超大字号
\pretitle{\begin{flushleft} \huge}
\posttitle{\end{flushleft}}
% 不要原作者,不要原作者
\preauthor{}
\postauthor{}
% 没有把日期放在标题正下方的打算
\predate{}
\postdate{}

% 常用色彩预设
\selectcolormodel{RGB}
\definecolor{ThemeDarkBlue}{RGB}{16, 27, 47}
\definecolor{ThemeYellow}{RGB}{255, 217, 202}
\definecolor{LightGrey}{RGB}{164, 164, 164}  % Grey - 英式拼写
\definecolor{LightGray}{RGB}{164, 164, 164}  % Gray - 美式拼写

\setcounter{section}{-1}  % 这条命令会让 section 序号从 0  开始。

\begin{document}

    \title{\textbf{COSSIG 文档样式标准}}
    \author{}  % 占位
    \date{}  % 占位

    \maketitle

    \section{关于本文档}

    文档编译时间:\today
    \\
    \\*
    本文档中的标准为 COSSIG (或称 COS-SIG,全称“中国开源兴趣小组”)社区文档的格式和样式标准。这些文档可能是贡献指南、贡献标准、公告或者说明书等。
    原则上,凡是 PDF 格式的社区文档,都最好能遵循这份标准。\\*
    {\color{Grey0} \sout{我们真的只是推荐你遵守,你不遵守我们也没办法拿你怎么办}}
    \\
    \\*
    本文档中的规定分为 3 个等级:\\*
    \begin{itemize}
        \item \textbf{\large{硬性规定}}\\*
        本文档中只有极少数的规定属于硬性规定。这些规定和本社区的性质和利益息息相关。
        例如:如果在一份官方文档中使用了微软雅黑作为中文字体,我们可能会被收取巨额的授权费用,所以“不要用微软雅黑”可能是硬性规定。
        \\
        \item \textbf{\large{推荐遵守的规定}}\\*
        这些规定通常涉及社区文档的可读性、便利性和视觉上的统一。例如:中文正文的字号为 10pt。
        \\
        \item \textbf{\large{不遵守也行的规定}}\\*
        这些规定实在太过于细枝末节,作为贡献者可以不用理会。(然而,作为本文档的编写者和本社区的美工,我会尽量让每份 PDF 文档遵循这里提到的全部规定 —— 注)
        \\
    \end{itemize}

    本文档的格式及样式遵守本标准;本文档及 LaTeX 源代码适用于“你他*想拿来干啥都行公共许可证”(WTFPL, Do What the F*ck You Want Public Licence)2.0 版本。


    \section{硬性规定}
    
    \textbf{请在社区文档中使用可以免费商用的字体及第三方美术素材,或者本社区自己生产的美术素材。}\\*
    虽然本社区不具备盈利性质,但我们希望能够避免可能的法务问题,因为我们真的完全没有任何钱可以交罚款。因此,我们强烈建议贡献者在提交贡献之前能够仔细检查所用素材的授权。
    \\
    \\*
    从工作流程上来说,在贡献被接受之前,社区美工也会帮你进行检查。违反此条规定不会导致任何惩罚,但是你会需要修改文档中使用的素材,直到它们符合此条规定为止。


    \section{推荐遵守的规定}
    目前而言,可以通过在你的 TeX 文件中引入 \ttfamily documentation-styles.tex\rmfamily 来将文档格式化成合乎以下规定的形式。\\*

    \begin{itemize}
        \item\textbf{\large{纸张尺寸及朝向}}\\*
        社区文档使用 A4 尺寸,竖向。
        \\
        \item\textbf{\large{字体及字号}}\\*
        中文正文:思源黑体(大陆简体),10 pt\\*
        英文正文:Source Sans Pro,10pt\\*
        英文等宽(代码用途):\ttfamily Source Code Pro,10pt \rmfamily\\*
        \\
    \end{itemize}    

\end{document}
