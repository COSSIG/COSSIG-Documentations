% 这个文件包含了合规的文档样式。如果想要让你的文档的样式符合规定,请在你的 tex 代码中引用这个文件。

\documentclass{article}[10pt, a4paper]

\usepackage{geometry}
[
    left=25mm, right=25mm, top=25mm, bottom=25mm, headheight=18pt
]

\usepackage{setspace}
\usepackage{hyphenat}
\usepackage{datetime2}
\usepackage{fontspec}
\usepackage{parskip}
\usepackage[x11names]{xcolor}
\usepackage[normalem]{ulem}
\usepackage{titling}
\usepackage{titlesec}

\usepackage{xeCJK}
\xeCJKsetup{CJKspace=true}  % 编译时不忽视中、日、韩文文字间的空格

% 正文行间距设定
\doublespacing{}

% 正文字体 - “思源”系列
\setCJKmainfont{Source Han Sans SC}  % 中文字体:思源黑体(大陆简体)
\setmainfont{Source Sans Pro}

% 等宽字体(本来想跟前两个同一个系列的,但是 Source Code Pro 实在太丑了……)
\setmonofont{IBM Plex Mono}

% 特殊用途字体 - 用于序号和特别标题

\newfontfamily\Montserrat{Montserrat-Regular}
[
    ItalicFont = Montserrat-Italic,
    BoldFont = Montserrat-Bold,
    BoldItalicFont = Montserrat-BoldItalic
]

\newfontfamily\MontserratExb{Montserrat-ExtraBold}
[
    ItalicFont = Montserrat-ExtraBoldItalic,
    BoldFont = Montserrat-Black,
    BoldItalicFont = Montserrat-BlackItalic
]

% 常用色彩预设
\selectcolormodel{RGB}
\definecolor{ThemeDarkBlue}{RGB}{16, 27, 47}
\definecolor{ThemeYellow}{RGB}{255, 217, 202}
\definecolor{LightGrey}{RGB}{164, 164, 164}  % Grey - 英式拼写
\definecolor{LightGray}{RGB}{164, 164, 164}  % Gray - 美式拼写

% 标题设定 - 大标题
% 左对齐标题;超大字号
\pretitle{\begin{flushleft} \huge \color{ThemeDarkBlue}}
\posttitle{\end{flushleft}}
% 不要原作者,不要原作者
\preauthor{}
\postauthor{}
% 没有把日期放在标题正下方的打算
\predate{}
\postdate{}

% 标题设定 - 小节标题
% 不要在下面的代码里乱加空格或者注释,可能会让标题序号前面无缘无故多出空格
% 以上提到的行为可能跟 TeX 编译器处理空格的方式有关系,但反正不太妙就是了

\renewcommand{\thesection}
{% 序号设定 - 字体为 Montserrat ExtraBold
    \MontserratExb\arabic{section}/
}  

\titleformat{\section}[block]
    {% 大号字体,行间距 1.2,深蓝色
        \Large
        \setstretch{1.2}
        \color{ThemeDarkBlue}
    }
    {% 序号和标题间换行
        \thesection\\*
    }
    {0cm}
    {\textbf}  % 标题本体使用粗体
