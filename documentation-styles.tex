% 这个文件包含了合规的文档样式。如果想要让你的文档的样式符合规定,请在你的 tex 代码中引用这个文件。

\documentclass[10pt, a4paper]{article}

\usepackage[left=25mm, right=25mm, top=25mm, bottom=25mm, headheight=18pt]{geometry}

\usepackage{setspace}
\usepackage{hyphenat}
\usepackage{datetime2}
\usepackage{fontspec}
\usepackage{parskip}
\usepackage[x11names]{xcolor}
\usepackage[normalem]{ulem}
\usepackage{titling}

\usepackage{xeCJK}
\xeCJKsetup{CJKspace=true}  % 编译时不忽视中、日、韩文文字间的空格

% 正文字体 - “思源”系列
\setCJKmainfont{Source Han Sans SC}  % 中文字体:思源黑体(大陆简体)
\setmainfont{Source Sans Pro}
\setmonofont{Source Code Pro}

% 标题设定
% 左对齐标题;超大字号
\pretitle{\begin{flushleft} \huge}
\posttitle{\end{flushleft}}
% 不要原作者,不要原作者
\preauthor{}
\postauthor{}
% 没有把日期放在标题正下方的打算
\predate{}
\postdate{}

% 常用色彩预设
\selectcolormodel{RGB}
\definecolor{ThemeDarkBlue}{RGB}{16, 27, 47}
\definecolor{ThemeYellow}{RGB}{255, 217, 202}
\definecolor{LightGrey}{RGB}{164, 164, 164}  % Grey - 英式拼写
\definecolor{LightGray}{RGB}{164, 164, 164}  % Gray - 美式拼写
